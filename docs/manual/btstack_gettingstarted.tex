\documentclass[a4paper,titlepage,oneside,12pt]{amsart} %amsart
\usepackage{graphicx}
\usepackage{hyperref}
%\usepackage{geometry} % see geometry.pdf on how to lay out the page. There's lots.
\usepackage[margin=1.3in]{geometry}
\geometry{a4paper} % or letter or a5paper or ... etc
% \geometry{landscape} % rotated page geometry
\usepackage{ifthen}

% See the ``Article customise'' template for come common customisations
\usepackage[usenames,dvipsnames]{color}

\usepackage[table]{xcolor}

\definecolor{lightgray}{RGB}{245, 245, 245}
\definecolor{bkblue}{RGB}{18, 47, 76}
\definecolor{bklightblue}{RGB}{102, 131, 158}
\definecolor{mygreen}{rgb}{0,0.6,0}
\definecolor{orange}{RGB}{255,153,0}

\usepackage{opensans}
\usepackage{setspace}
\usepackage{booktabs} \newcommand{\ra}[1]{\renewcommand{\arraystretch}{#1}}

\usepackage{color}

% show todos
\newcommand{\todo}[1]{\colorbox{yellow}{#1}}
% ignore todos
% \newcommand{\todo}[1]{}

\newcommand{\toread}[1]{{\color{bklightblue} #1}}

\usepackage{listings}
\lstset{ %
float,
language=C,                % choose the language of the code
basicstyle=\footnotesize,       % the size of the fonts that are used for the code
%numbers=left,                   % where to put the line-numbers
%numberstyle=\footnotesize,      % the size of the fonts that are used for the line-numbers
%stepnumber=1,                   % the step between two line-numbers. If it is 1 each line will be numbered
%numbersep=5pt,                  % how far the line-numbers are from the code
backgroundcolor=\color{lightgray},  % choose the background color. You must add \usepackage{color}
showspaces=false,               % show spaces adding particular underscores
showstringspaces=false,         % underline spaces within strings
showtabs=false,                 % show tabs within strings adding particular underscores
frame=single,           % adds a frame around the code
framerule=0.2pt,
tabsize=2,          % sets default tabsize to 2 spaces
captionpos=b,           % sets the caption-position to bottom
breaklines=true,        % sets automatic line breaking
breakatwhitespace=false,    % sets if automatic breaks should only happen at whitespace
escapeinside={\%*}{*)},          % if you want to add a comment within your code
belowcaptionskip=5em,
belowskip=1em,
aboveskip=1.8em,
commentstyle=\itshape\color{mygreen},
keywordstyle=\bfseries\color{black},
identifierstyle=\color{black},
stringstyle=\color{blue},
morekeywords={*, timer_source_t, data_source_t, uint32_t, uint16_t, uint8_t, RUN_LOOP_TYPE, le_command_status_t, gatt_client_t,
 							le_characteristic_t, le_service_t, le_characteristic_descriptor_t, service_record_item_t, bd_addr_t, btstack_packet_handler_t,
							hci_cmd_t, bt_control_t, remote_device_db_t, link_key_t, device_name_t, hci_transport_t, hci_uart_config_t, sdp_query_event_t,
							sdp_query_complete_event_t, sdp_query_rfcomm_service_event_t, sdp_parser_event_t, sdp_parser_event_type_t,
							sdp_parser_attribute_value_event_t, sdp_parser_complete_event_t, advertising_report_t, gc_state_t, le_service_event_t,
							le_characteristic_event_t}
}

% Bluetopia & TI MSP430 + Stellaris
% http://processors.wiki.ti.com/index.php/CC256x_Bluetopia_Stack#Demos

% Setup MSP430+PAN1315
% 
\newcommand{\versionNr}{1.3}
\newcommand{\authorMila}{Dr. sc. Milanka Ringwald}
\newcommand{\authorMatthias}{Dr. sc. Matthias Ringwald}
\newcommand{\bkContact}{\href{contact@bluekitchen-gmbh.com}{contact@bluekitchen-gmbh.com}}
\newcommand{\barWidth}{0.3cm}
\newcommand{\urlfoot}[2]{\href{#1}{{\color{blue} #2}}\footnote{#1}}

\makeatletter
\renewcommand{\maketitle}{
  \begin{titlepage}
    \fosfamily
    \begin{center}
    \begin{minipage}[b]{\textwidth}
        \begin{minipage}[b]{.1\textwidth}
            \color{bkblue}\rule{\barWidth{}}{22cm}
        \end{minipage}
        \hfill\begin{minipage}[b]{.8\textwidth}\begin{flushright}
            {\color{bkblue}
%            VERSION \versionNr{} \\
            \today \\}
            \vspace*{7.5cm}
            \hfill\includegraphics[width=0.85\textwidth]{picts/bklogo.pdf}
            \vspace*{1.5cm}
            \begin{spacing}{2} 
                {\huge \color{bkblue} \@title} \\ 
                {\Large \color{bklightblue} Including Quickstart Guide}   
            \end{spacing} 
            \vspace*{1.5cm}
            {\color{bkblue}\large \authorMila \\
            \large \authorMatthias \\
            \large \bkContact\\ }
        \end{flushright}\end{minipage}
        \vfill
        \begin{minipage}[b]{\textwidth}
            \color{bklightblue}\rule{\barWidth{}}{\barWidth{}}
        \end{minipage}
    \end{minipage}

    
    \end{center}
  \end{titlepage}
}
\makeatother

\title[BTstack Manual] {BTstack Manual} 
\author{Copyright \copyright 2012-2015 BlueKitchen GmbH}

%%% BEGIN DOCUMENT

\newcommand{\UserGuide}{\urlfoot{http://processors.wiki.ti.com/index.php/PAN1315EMK\_User\_Guide\#RF3\_Connector}{User Guide}{}}
\newcommand{\MSPGCCWiki}{\urlfoot{http://sourceforge.net/apps/mediawiki/mspgcc/index.php?title=MSPGCC\_Wiki}{MSPGCC Wiki}}
\newcommand{\GNUMake}{\urlfoot{http://gnuwin32.sourceforge.net/packages/make.htm}{GNU Make}}
\newcommand{\Python}{\urlfoot{http://www.python.org/getit/}{Python}}
\newcommand{\mspgcc}{\urlfoot{http://sourceforge.net/projects/mspgcc/files/Windows/mingw32/}{mspgcc}}
\newcommand{\BTSfile}{\urlfoot{http://processors.wiki.ti.com/index.php/CC256x\_Downloads}{BTS file}}
\newcommand{\MSPFlasher}{\urlfoot{http://processors.wiki.ti.com/index.php/MSP430\_Flasher\_-\_Command\_Line\_Programmer}{MSP430Flasher}}
\newcommand{\MSPDebug}{\urlfoot{http://mspdebug.sourceforge.net/}{MSPDebug}}
\newcommand{\BtstackGithub}{\urlfoot{https://github.com/bluekitchen/btstack/archive/master.zip}{BTstack's page}}
\newcommand{\gccarm}{\urlfoot{https://launchpad.net/gcc-arm-embedded}{arm-gcc}}
\newcommand{\OpenOCD}{\urlfoot{http://openocd.org}{OpenOCD}}
\newcommand{\mplabxc}{\urlfoot{http://www.microchip.com/pagehandler/en\_us/devtools/mplabxc/}{MPLAB XC}}
\newcommand{\PICkit}{\urlfoot{http://www.microchip.com/DevelopmentTools/ProductDetails.aspx?PartNO=pg164130}{PICkit 3}}

%level -1: part, 0: chapter, 1: section, etc.
\setcounter{tocdepth}{3}

\begin{document}

\maketitle

\tableofcontents
\pagebreak


Thanks for checking out BTstack! In this manual, we first provide the usual 'quick starter guide' before highlighting BTstack's main design choices and going into more details with a few examples. Finally, we outline the basic steps when integrating BTstack into existing single-threaded or even multi-threaded environments. The Revision History is shown in the Appendix \ref{appendix:revision_history} on page \pageref{appendix:revision_history}.

\section{Quick Start}

\subsection{General Tools}
On Unix-based systems, git, make, and Python are usually installed. If not, use the system's packet manager to install them. 

On Windows, you need to install and configure mspgcc, GNU Make, Python, and optionally git manually:
 \begin{itemize}
 \item \GNUMake{} for Windows: Add its bin folder to the Windows Path in Environment Variables. The bin folder is where make.exe resides, and it's usually located in \path{C:\Program Files\GnuWin32\bin}. 
 \item \Python{} for Windows: Add Python installation folder to the Windows Path in Environment Variables.
 \end{itemize}

Adding paths to the Windows Path variable:
\begin{itemize} \label{sec:windowsPath}
   \item Go to: Control Panel$\rightarrow$System$\rightarrow$Advanced tab$\rightarrow$Environment Variables.
   \item The top part contains a list of User variables. 
   \item Click on the Path variable and then click edit.
   \item Go to the end of the line, then append the path to the list., for example, \path{C:\mspgcc\bin} for mspgcc.
   \item Ensure that there is a semicolon before and after \path{C:\mspgcc\bin}. 
\end{itemize}

\subsection{Getting BTstack from GitHub}

Use git to clone the latest version:
	\begin{lstlisting}
 		git clone https://github.com/bluekitchen/btstack.git
	\end{lstlisting}
Alternatively, you can download it as a ZIP archive from \BtstackGithub{}on GitHub.

\subsection{Compiling the examples and loading firmware}
This step is platform specific. To compile and run the examples, you need to download and install the platform specific toolchain and a flash tool. For TI's CC256x chipsets, you also need the correct init script, or "Service Pack" in TI nomenclature. Assuming that these are provided, go to \path{btstack/platforms/$PLATFORM$} folder in command prompt and run make. If all the paths are correct, it will generate several firmware files. These firmware files can be loaded onto the device using platform specific flash programmer. For the PIC32-Harmony platform, a project file for the MPLAB X IDE is provided, too.

\begin{table*}\centering
\caption{Overview of platform specific toolchains, programmers, and used chipsets.}

\resizebox{\textwidth}{!}{\begin{minipage}{\textwidth}
\rowcolors{1}{lightgray}{white}

\begin{tabular}{p{4cm}p{2cm}p{3cm}p{4cm} }
\toprule
\hiderowcolors Platform & Chipset & Toolchain & Programmer\\ \showrowcolors
\midrule
ez430-rf2560, msp-exp430f5438,  msp430f5229lp & CC256x & \mspgcc{} &\MSPFlasher{},  \MSPDebug{} \\           
stm32-f103rb-nucleo     & CC256x   & \gccarm{}    & \OpenOCD{} \\      
pic32-harmony               & CSR8811 & \mplabxc{} & \PICkit{} \\ \hiderowcolors
\bottomrule
 \label{table:platformCompiler}
 \end{tabular}
 
 \end{minipage} }
\end{table*}

\subsection{Run the Example}

As a first test, we recommend the SPP Counter example (see Section \ref{section:sppcounter}). During the startup, for TI chipsets, the init script is transferred, and the Bluetooth stack brought up. After that, the development board is discoverable as "BTstack SPP Counter" and provides a single virtual serial port. When you connect to it, you'll receive a counter value as text every second.

% The SPP Counter doesn't use the display to keep the memory footprint small.
%  The HID demo has a fancier user interface - it uses a display to show the discovery process and connection establishment with a Bluetooth keyboard, as well as the text as you type.
 
\subsection{Platform specifics}
 
% !TEX root = btstack_gettingstarted.tex

In the following, we provide more information on specific platform setups, toolchains, programmers, and init scripts.

\subsubsection{Texas Instruments MSP430-based boards}

\emph{Compiler Setup} The MSP430 port of BTstack is developed using the Long Term Support (LTS) version of mspgcc. General information about it and installation instructions are provided on the \MSPGCCWiki{}. On Windows, you need to download and extract \mspgcc{} to \path{C:\mspgcc}. Add \path{C:\mspgcc\bin} folder to the Windows Path in Environment variable as explained in Section \ref{sec:windowsPath}.

\emph{Loading Firmware} To load firmware files onto the MSP430 MCU for the MSP-EXP430F5438 Experimeneter board, you need a programmer like the MSP430 MSP-FET430UIF debugger or something similar. The eZ430-RF2560 and MSP430F5529LP contain a basic debugger. the Now, you can use one of following software tools:

 \begin{itemize}
 \item  \MSPFlasher{} (windows-only):
 	\begin{itemize}
 	   \item Use the following command, where you need to replace the \path{BINARY_FILE_NAME.hex} with the name of your application:
	\end{itemize} 
\end{itemize}
	
	   \begin{lstlisting}
 		MSP430Flasher.exe -n MSP430F5438A -w "BINARY_FILE_NAME.hex" -v -g -z [VCC]
	   \end{lstlisting}

 \begin{itemize}
	
	\item \MSPDebug{}: An example session with the MSP-FET430UIF connected on OS X is given in following listing:
\end{itemize}

\begin{lstlisting}
mspdebug -j -d /dev/tty.FET430UIFfd130 uif
... 
prog blink.hex
run
\end{lstlisting}

\subsubsection{Texas Instruments CC256x-based chipsets}
\emph{CC256x Init Scripts} In order to use the CC256x chipset on the PAN13xx modules and others, an initialization script must be obtained. Due to licensing restrictions, this initialization script must be obtained separately as follows:
\begin{itemize}
\item Download the \BTSfile{} for your CC256x-based module.
\item Copy the included .bts file into \path{btstack/chipset-cc256x}
\item In \path{chipset-cc256x}, run the Python script: $./convert\_bts\_init\_scripts.py$
\end{itemize}

The common code for all CC256x chipsets is provided by $bt\_control\_cc256x.c$. During the setup, $bt\_control\_cc256x\_instance$ function is used to get  a $bt\_control\_t$ instance and passed to $hci\_init$ function. 

Note: Depending on the CC256x-based module you're using, you'll need to update the reference \path{bluetooth_init_cc256...} in the Makefile to match the downloaded file.

Update: For the latest revision of the CC256x chipsets, the CC2560B and CC2564B, TI decided to split the init script into a main part and the BLE part. The conversion script has been updated to detect $blueooth\_init\_cc256x\_1.2.bts$ and adds $BLE\_init\_cc256x\_1.2.bts$ if present and merges them into a single .c file.

\subsubsection{MSP-EXP430F5438 + CC256x Platform}
\label{platform:msp430}
\emph{Hardware Setup} We assume that a PAN1315, PAN1317, or PAN1323 module is plugged into RF1 and RF2 of the MSP-EXP430F5438 board and the "RF3 Adapter board" is used or at least simulated. See \UserGuide{}. 

\subsubsection{STM32F103RB Nucleo + CC256x Platform}
To try BTstack on this platform, you'll need a simple adaptor board. For details, please read the documentation in \path{platforms/stm32-f103rb-nucleo/README.md}.

\subsubsection{PIC32 Bluetooth Audio Development Kit}
The PIC32 Bluetooth Audio Development Kit comes with the CSR8811-based BTM805 Bluetooth module. In the port, the UART on the DAC daugtherboard was used for the debug output. Please remove the DAC board and connect a 3.3V USB-2-UART converter to GND and TX to get the debug output.

In \path{platforms/pic32-harmony}, a project file for the MPLAB X IDE is provided as well as a regular Makefile. Both assume that the MPLAB XC32 compiler is installed. The project is set to use -Os optimization which will cause warnings if you only have the Free version. It will still compile a working example. For this platform, we only provide the SPP and LE Counter example directly. Other examples can be run by replacing the spp\_and\_le\_counter.c file with one of the other example files.

\subsubsection{libusb}
The quickest way to try BTstack is on a Linux or OS X system with an additional USB Bluetooth module. The Makefile in \path{platforms/libusb}
requires pkgconfig and libusb-1.0 or higher to be installed. 

On Linux, it's usually necessary to run the examples as root as the kernel needs to detach from the USB module.

On OS X, it's neccessary to tell the OS to only use the internal Bluetooth. For this, execute
\begin{lstlisting}
sudo nvram bluetoothHostControllerSwitchBehavior=never
\end{lstlisting}

It's also possible to run the examples on Win32 systems. For this, 
\begin{itemize}
\item Install MSYS and MINGW32 using the MINGW installer
\item Compile and install libusb-1.0.19 to /usr/local/ in msys command shell
\item Setup a USB Bluetooth dongle for use with libusb-1.0:
	\begin{itemize}
	\item Start Zadig from http://zadig.akeo.ie
	\item Select Options $\rightarrow$ List all devices
	\item Select USB Bluetooth dongle in the big pull down list
	\item Select WinUSB (libusb) in the right pull pull down list
	\item Select Replace Driver
	\end{itemize}
\end{itemize}
Now, you can run the examples from the msys shell the same way as on Linux/OS X.




\section{BTstack Architecture}

As well as any other communication stack, BTstack is a collection of state machines that interact with each other. There is one or more state machines for each protocol and service that it implements. The rest of the architecture follows these fundamental design guidelines:

\begin{itemize}
\item \emph{Single threaded design} - BTstack does not use or require multi-threading to handle data sources and timers. Instead, it uses a single run loop.
\item \emph{No blocking anywhere} - If Bluetooth processing is required, its result will be delivered as an event via registered packet handlers.
\item \emph{No artificially limited buffers/pools} - Incoming and outgoing data packets are not queued.
\item \emph{Statically bounded memory (optionally)} - The number of maximum connections/channels/services can be configured.
\end{itemize}

\begin{figure}[htbp] %  figure placement: here, top, bottom, or page
   \centering
   \includegraphics[width=\textwidth]{picts/btstack-architecture.pdf} 
   \caption{BTstack-based single-threaded application. The Main Application contains the application logic, e.g., reading a sensor value and providing it via the Communication Logic as a SPP Server.
    The Communication Logic is often modeled as a finite state machine with events and data coming from either the Main Application or from BTstack via registered  packet handlers (PH).   
    BTstack's Run Loop is responsible for providing timers and processing incoming data.
}
    
   \label{fig:BTstackArchitecture}
\end{figure}

Figure \ref{fig:BTstackArchitecture} shows the general architecture of a BTstack-based application that includes the BTstack run loop.

\subsection{Single threaded design}
BTstack does not use or require multi-threading. It uses a single run loop to handle data sources and timers. Data sources represent communication interfaces like an UART or an USB driver. 
Timers are used by BTstack to implement various Bluetooth-related timeouts. For example, to disconnect a Bluetooth baseband channel without an active L2CAP channel after 20 seconds. They can also be used to handle periodic events. During a run loop cycle, the callback functions of all registered data sources are called. Then, the callback functions of timers that are ready are executed. 

For adapting BTstack to multi-threaded environments, see Section \ref{section:multithreading}.


\subsection{No blocking anywhere}

Bluetooth logic is event-driven. Therefore, all BTstack functions are non-blocking, i.e., all functions that cannot return immediately implement an asynchronous pattern. If the arguments of a function are valid, the necessary commands are sent to the Bluetooth chipset and the function returns with a success  value. The actual result is delivered later as an asynchronous event via registered packet handlers. 

If a Bluetooth event triggers longer processing by the application, the processing should be split into smaller chunks. The packet handler could then schedule a timer that manages the sequential execution of the chunks.

\subsection{No artificially limited buffers/pools}
Incoming and outgoing data packets are not queued. BTstack delivers an incoming data packet to the application before it receives the next one from the Bluetooth chipset. Therefore, it relies on the link layer of the Bluetooth chipset to slow down the remote sender when needed. 

Similarly, the application has to adapt its packet generation to the remote receiver for outgoing data. 
L2CAP relies on ACL flow control between sender and receiver. If there are no free ACL buffers in the Bluetooth module, the application cannot send.  For RFCOMM, the mandatory credit-based flow-control limits the data sending rate additionally. The application can only send an RFCOMM packet if it has RFCOMM credits.

\subsection{Statically bounded memory}
BTstack has to keep track of services and active connections on the various protocol layers. The number of maximum connections/channels/services can be configured. In addition, the non-persistent database for remote device names and link keys needs memory and can be be configured, too. These numbers determine the amount of static memory allocation.


\section{How to use BTstack}
BTstack implements a set of basic Bluetooth protocols. To make use of these to connect to other devices or to provide own services, BTstack has to be properly configured during application startup. 

In the following, we provide an overview of the memory management, the run loop, and services that are necessary to setup BTstack. From the point when the run loop is executed, the application runs as a finite state machine, which processes events received from BTstack. BTstack groups events logically and provides them over packet handlers, of which an overview is provided here. Finally, we describe the RFCOMM credit-based flow-control, which may be necessary for resource-constraint devices. Complete examples for the MSP430 platforms will be presented in Chapter \ref{examples}. 

\subsection{Memory configuration}
\label{section:memory_configuration}

The structs for services, active connections and remote devices can be allocated in two different manners: 
        \begin{itemize}
        \item statically from an individual memory pool, whose maximal number of elements  is defined in the config file. To initialize the static pools, you need to call \emph{btstack\_memory\_init} function. An example of memory configuration for a single SPP service with a minimal L2CAP MTU is shown in Listing \ref{memoryConfigurationSPP}.
        \item dynamically using the \emph{malloc/free} functions, if HAVE\_MALLOC is defined in config file. 
        \end{itemize}

If both HAVE\_MALLOC and  maximal size of a pool are defined in the config file, the statical allocation will take precedence. In case that both are omitted, an error will be raised.

The memory is set up by calling \emph{btstack\_memory\_init} function:
\begin{lstlisting}
btstack_memory_init();
\end{lstlisting}


\subsection{Run loop}
\label{section:run_loop}

BTstack uses a run loop to handle incoming data and to schedule work. The run loop handles events from two different types of sources: data sources and timers. Data sources represent communication interfaces like an UART or an USB driver. Timers are used by BTstack to implement various Bluetooth-related timeouts. They can also be used to handle periodic events. 

Data sources and timers are represented by the \emph{data\_source\_t} and \emph{timer\_source\_t} structs respectively. Each of these structs contain a linked list node and a pointer to a callback function. All active timers and data sources are kept in link lists. While the list of data sources is unsorted, the timers are sorted by expiration timeout for efficient processing.

The complete run loop cycle looks like this: first, the callback function of all registered data sources are called in a round robin way. Then, the callback functions of timers that are ready are executed. Finally, it will be checked if another run loop iteration has been requested by an interrupt handler. If not, the run loop will put the MCU into sleep mode.

Incoming data over the UART, USB, or timer ticks will generate an interrupt and wake up the microcontroller. In order to avoid the situation where a data source becomes ready just before the run loop enters sleep mode, an interrupt-driven data source has to call the \emph{embedded\_trigger} function. The call to \emph{embedded\_trigger} sets an internal flag that is checked in the critical section just before entering sleep mode. 

Timers are single shot: a timer will be removed from the timer list before its event handler callback is executed. If you need a periodic timer, you can re-register the same timer source in the callback function, as shown in Listing \ref{PeriodicTimerHandler}. Note that BTstack expects to get called periodically to keep its time, see Section \ref{section:timeAbstraction} for more on the tick hardware abstraction. 

The Run loop API is provided in Appendix \ref{appendix:api_run_loop}. To enable the use of timers, make sure that you defined HAVE\_TICK in the config file.

In your code, you'll have to configure the run loop before you start it as shown in Listing \ref{RunLoopExecution}. The application can register data sources as well as timers, e.g., periodical sampling of sensors, or communication over the UART.

The run loop is set up by calling \emph{run\_loop\_init} function for embedded systems:
\begin{lstlisting}
run_loop_init(RUN_LOOP_EMBEDDED);
\end{lstlisting}


\begin{lstlisting}[float, caption=Periodic counter, label=PeriodicTimerHandler]
#define TIMER_PERIOD_MS 1000
timer_source_t periodic_timer;

void register_timer(timer_source_t *timer, uint16_t period){
    run_loop_set_timer(timer, period);
    run_loop_add_timer(timer);
}

void timer_handler(timer_source_t *ts){
    // do something,
    ... e.g., increase counter,
    
    // then re-register timer
    register_timer(ts, TIMER_PERIOD_MS);
} 

void timer_setup(){
    // set one-shot timer
    run_loop_set_timer_handler(&periodic_timer, &timer_handler);
    register_timer(&periodic_timer, TIMER_PERIOD_MS);
}
\end{lstlisting}

\subsection{BTstack initialization}
\label{section:btstack_initialization}
To initialize BTstack you need to initialize the memory and the run loop as explained in Sections \ref{section:memory_configuration} and \ref{section:run_loop} respectively, then setup HCI and all needed higher level protocols.

The HCI initialization has to adapt BTstack to the used platform and requires four arguments. These are:
\begin{itemize}
\item \emph{Bluetooth hardware control}: The Bluetooth hardware control API can provide the HCI layer with a custom initialization script, a vendor-specific baud rate change command, and system power notifications. It is also used to control the power mode of the Bluetooth module, i.e., turning it on/off and putting to sleep. In addition, it provides an error handler \emph{hw\_error} that is called when a Hardware Error is reported by the Bluetooth module. The callback allows for persistent logging or signaling of this failure. 

Overall, the struct \emph{bt\_control\_t} encapsulates common functionality that is not covered by the Bluetooth specification. As an example, the \emph{bt\_con-trol\_cc256x\_in-stance} function returns a pointer to a control struct suitable for the CC256x chipset.
\end{itemize}

\begin{lstlisting}
bt_control_t * control = bt_control_cc256x_instance();
\end{lstlisting}

\begin{itemize}
\item \emph{HCI Transport implementation}: On embedded systems, a Bluetooth module can be connected via USB or an UART port. BTstack implements two UART based protocols: HCI UART Transport Layer (H4) and H4 with eHCILL support, a lightweight low-power variant by Texas Instruments.
These are accessed by linking the appropriate file (\path{src/hci_transport_h4_dma.c} resp. \path{src/hci_transport_h4_ehcill_dma.c)} and then getting a pointer to HCI Transport implementation. For more information on adapting HCI Transport to different environments, see Section \ref{section:hci_transport}.
\end{itemize}

\begin{lstlisting}
hci_transport_t * transport = hci_transport_h4_dma_instance();
\end{lstlisting}

\begin{itemize}
\item \emph {HCI Transport configuration}: As the configuration of the UART used in the H4 transport interface are not standardized, it has to be provided by the main application to BTstack. In addition to the initial UART baud rate, the main baud rate can be specified. The HCI layer of BTstack will change the init baud rate to the main one after the basic setup of the Bluetooth module. A baud rate change has to be done in a coordinated way at both HCI and hardware level. First, the HCI command to change the baud rate is sent, then it is necessary to wait for the confirmation event from the Bluetooth module. Only now, can the UART baud rate changed. As an example, the CC256x has to be initialized at 115200 and can then be used at higher speeds.
\end{itemize}

\begin{lstlisting}
hci_uart_config_t* config = hci_uart_config_cc256x_instance();
\end{lstlisting}

\begin{itemize}
\item \emph {Persistent storage} - specifies where to persist data like link keys or remote device names. This commonly requires platform specific code to access the MCU's EEPROM of Flash storage. For the first steps, BTstack provides a (non) persistent store in memory. For more see Section \ref{section:persistent_storage}.
\end{itemize}

\begin{lstlisting}
remote_device_db_t * remote_db = &remote_device_db_memory;
\end{lstlisting}


After these are ready, HCI is initialized like this:
\begin{lstlisting}
hci_init(transport, config, control, remote_db);
\end{lstlisting}

The higher layers only rely on BTstack and are initialized by calling the respective \emph{*\_init} function. These init functions register themselves with the underlying layer. In addition, the application can register packet handlers to get events and data as explained in the following section.


\subsection{Services}
One important construct of BTstack is \emph{service}. A service represents a server side component that handles incoming connections. So far, BTstack provides L2CAP and RFCOMM services. An L2CAP service handles incoming connections for an L2CAP channel and is registered with its protocol service multiplexer ID (PSM). Similarly,  an RFCOMM service handles incoming RFCOMM connections and is registered with the RFCOMM channel ID. Outgoing connections require no special registration, they are created by the application when needed. 

\begin{lstlisting}[float, caption=Memory configuration for an SPP service with a minimal L2CAP MTU., label=memoryConfigurationSPP]
#define HCI_ACL_PAYLOAD_SIZE 52
#define MAX_SPP_CONNECTIONS 1
#define MAX_NO_HCI_CONNECTIONS MAX_SPP_CONNECTIONS
#define MAX_NO_L2CAP_SERVICES  2
#define MAX_NO_L2CAP_CHANNELS  (1+MAX_SPP_CONNECTIONS)
#define MAX_NO_RFCOMM_MULTIPLEXERS MAX_SPP_CONNECTIONS
#define MAX_NO_RFCOMM_SERVICES 1
#define MAX_NO_RFCOMM_CHANNELS MAX_SPP_CONNECTIONS
#define MAX_NO_DB_MEM_DEVICE_NAMES  0
#define MAX_NO_DB_MEM_LINK_KEYS  3
#define MAX_NO_DB_MEM_SERVICES 1
\end{lstlisting}

\subsection{Where to get data - packet handlers}
\label{section:packetHandlers}

After the hardware and BTstack are set up, the run loop is entered. From now on everything is event driven. The application calls BTstack functions, which in turn may send commands to the Bluetooth module. The resulting events are delivered back to the application. Instead of writing a single callback handler for each possible event (as it is done in some other Bluetooth stacks), BTstack groups events logically and provides them over a single generic interface.  Appendix \ref{appendix:api_events_and_errors} summarizes the parameters and event codes of L2CAP and RFCOMM events, as well as possible errors and the corresponding error codes.

Here is summarized list of packet handlers that an application might use:
\begin{itemize}
\item HCI packet handler - handles HCI and general BTstack events if L2CAP is not used (rare case). 
\item L2CAP packet handler - handles HCI and general BTstack events.
\item L2CAP service packet handler - handles incoming L2CAP connections, i.e., channels initiated by the remote.
\item L2CAP channel packet handler - handles outgoing L2CAP connections,  i.e., channels initiated internally.
\item RFCOMM packet handler - handles RFCOMM  incoming/outgoing events and data.
\end{itemize}

\begin{table*}\centering
\caption{Functions for registering packet handlers}
\begin{tabular}{rl}\toprule
Packet Handler & Registering Function\\ 
\midrule
HCI packet handler &  \emph{hci\_register\_packet\_handler}\\
L2CAP packet handler &  \emph{l2cap\_register\_packet\_handler}\\
L2CAP service packet handler & \emph{l2cap\_register\_service\_internal}\\
L2CAP channel packet handler & \emph{l2cap\_create\_channel\_internal}\\
RFCOMM packet handler & \emph{rfcomm\_register\_packet\_handler}\\
\bottomrule
 \label{table:registeringFunction}
 \end{tabular}
\end{table*}

These handlers are registered with the functions listed in Table \ref{table:registeringFunction}. 

HCI and general BTstack events are delivered to the packet handler specified by \emph{l2cap\_register\_packet\_handler} function, or \emph{hci\_register\_packet\_handler}, if L2CAP is not used. In L2CAP, BTstack discriminates incoming and outgoing connections, i.e., event and data packets are delivered to different packet handlers. Outgoing connections are used access remote services, incoming connections are used to provide services. For incoming connections, the packet handler specified by \emph{l2cap\_register\_service} is used. For outgoing connections, the handler provided by \emph{l2cap\_create\_channel\_internal} is used.
Currently, RFCOMM provides only a single packet handler specified by \emph{rfcomm\_register\_packet\_handler} for all RFCOMM connections, but this will be fixed in the next API overhaul.
 
The application can register a single shared packet handler for all protocols and services, or use separate packet handlers for each protocol layer and service. A shared packet handler is often used for stack initialization and connection management.

Separate packet handlers can be used for each L2CAP service and outgoing connection. For example, to connect with a Bluetooth HID keyboard, your application could use three packet handlers: one to handle HCI events during discovery of a keyboard registered by \emph{l2cap\_register\_packet\_handler}; one that will be registered to an outgoing L2CAP channel to connect to keyboard and to receive keyboard data registered by \emph{l2cap\_create\_channel\_internal}; after that keyboard can reconnect by itself. For this, you need to register L2CAP services for the HID Control and HID Interrupt PSMs using \emph{l2cap\_register\_service\_internal}. In this call, you'll also specify a packet handler to accept and receive keyboard data. 

\newcommand{\BluetoothSpecification}{\urlfoot{https://www.bluetooth.org/Technical/Specifications/adopted.htm}{Bluetooth Specification}}
\newcommand{\BluetoothSpecificationURL}{\href{https://www.bluetooth.org/Technical/Specifications/adopted.htm}{\color{blue} Bluetooth Specification}}

% !TEX root = btstack_gettingstarted.tex
\section{Protocols and Profiles}
\label{section:protocols_profiles}

BTstack implements following Bluetooh protocols: HCI, L2CAP, L2CAP-LE, RFCOMM, SDP, SMP,  and ATT, as well as three profiles: GATT, GAP and SPP, see Figure \ref{fig:BTstackProtocolArchitecture}. 

\begin{figure}[htbp] %  figure placement: here, top, bottom, or page
   \centering
   \includegraphics[width=0.7\textwidth]{picts/btstack-protocols.pdf} 
   \caption{BTstack Protocol Architecture}
   \label{fig:BTstackProtocolArchitecture}
\end{figure}

\subsection{HCI - Host Controller Interface}
The HCI protocol provides a command interface to the Bluetooth chipset. 

\subsection{L2CAP -  Logical Link Control and Adaptation Protocol}
The L2CAP protocol supports higher level protocol multiplexing and packet fragmentation.
%\section{Security Levels and L2AP}

\subsection{L2CAP LE - L2CAP Low Energy Protocol}
The L2CAP LE variant is optimized for connectionless data used by Bluetooth Low Energy devices. It is the base for the Attribute Protocol (ATT) of Bluetooth LE, which defines how to discover, read, and write attributes on a peer device. % , and the conveying of quality of service information. - L2CAP _can_ do this, but BTstack does not

\subsection{RFCOMM - Radio Frequency Communication Protocol}
The Radio frequency communication (RFCOMM) protocol provides emulation of serial ports over the L2CAP protocol.
and reassembly. 
\subsubsection{RFCOMM flow control}
\label{section:flowcontrol}
RFCOMM has a mandatory credit-based flow-control. This means that two devices that established RFCOMM connection, use credits to keep track of how many more RFCOMM data packets can be sent to each. If a device has no (outgoing) credits left, it cannot send another RFCOMM packet, the transmission must be paused. During the connection establishment, initial credits are provided. BTstack tracks the number of credits in both directions. If no outgoing credits are available, the RFCOMM send function will return an error, and you can try later. For incoming data, BTstack provides channels and services with and without automatic credit management via different functions to create/register them respectively. If the management of credits is automatic, the new credits are provided when needed relying on ACL flow control - this is only useful if there is not much data transmitted and/or only one physical connection is used. If the management of credits is manual, credits are provided by the application such that it can manage its receive buffers explicitly. 
\subsubsection{RFCOMM configuration - TODO}
\todo{rfcomm configuration}
%\subsection{add RFCOMM port configuration for both local and remote}
%\subsection{add RFCOMM modem and line status control/information}
%\subsection{add RFCOMM\_AGGREGATE\_FLOW\_OFF to recoverable RFCOMM send errors (example in tex, code)}
%\section{Security Levels and RFCOMM}

\subsection{SDP - Service Discovery Protocol}
The SDP protocol allows to discover services provided by a Bluetooth device. 

\subsection{SMP - Security Manager Protocol }
The SMP protocol allows to setup authenticated and encrypted connection.

\subsection{ATT - Attribute Protocol}
ATT is a wire application protocol, while GATT dictates how ATT is employed in service composition. Every Low Energy profile must be based on GATT. So, ultimately, every LE service uses ATT as the application protocol. An ATT server stores attributes. An ATT client stores nothing; it uses the ATT wire protocol to read and write values on server attributes. Most of the ATT protocol is pure client-server: client takes the initiative, server answers. But ATT has notification and indication capabilities, in which the server takes the initiative of notifying a client that an attribute value has changed, saving the client from having to poll the attribute.

\subsection{SPP - Serial Port Profile}
The SPP profile defines how to set up virtual serial ports and connect two Bluetooth enabled devices. See Appendix \ref{appendix:api_} for the SPP API.

\subsection{GAP - Generic Access Profile for Low Energy}
The GAP profile defines how to discover and how to connect to a Bluetooth device. There are several GAP roles that a Bluetooth device can take, but the most important ones are the Central and the Peripheral role. Peripheral devices are those that provide information or can be controlled and central devices are those that consume information or control the peripherals. Before the connection can be established, devices are first going through the advertising process. What happens with the peripheral device after the central device connect to a it, depends on the peripheral's Bluetooth controller. The peripheral will either stop advertising itself and other devices will no longer be able to see it or connect to it until the existing connection is broken, or it will be able to continue with advertising so that the parallel connections can be established.
\subsubsection{GAP BLE Roles}
There are four GAP roles defined for a Bluetooth low energy device: Broadcaster, Observer, Peripheral and Central. A device may operate in multiple GAP roles concurrently.
\begin{itemize}
\item \emph{GAP Broadcaster Role} - A broadcast device only sends advertisements and cannot be connected. It can emit some useful data as part of the advertisement. The most prominent use for this is Apple's iBeacon technology which uses broadcast devices to emit a unique ID. Apple's iOS framework then help to map this ID onto a specific location, e.g., in a museum. Broadcasting is efficient as no connection and no ATT database are needed. To control energy consumption the broadcast interval can be configured. An advertisement can contain up to 31 bytes of information. In addition, another 31 bytes of information can be sent in the scan response.
\item \emph{GAP Observer Role} - An observer device only receives advertising events and cannot be connected. 
\item \emph{GAP Central Role} - The role of the central device is to scan for peripherals, connect to them, and discover and receive data from them or sends data to control them. During scanning the central device can retrieve information on other device such are its name and unique number, as well as some broadcast data from its services. Upon connection, the central explores the device by discovering its primary and included services, characteristics, and characteristic descriptors.
\item \emph{GAP Peripheral Role} - The role of a peripheral device is to deliver information on their inputs, i.e. sensor values, battery level, current time, to the applications running on central devices. It can also receive a write request from a central device and control connected actors, e.g. turn on and set the color of the light. Peripherals can broadcast data, they can be discovered and connected to by a central device, they can stay also disconnected and then establish connection when needed. 
\end{itemize} 
\subsubsection{Advertising and Scan Response Data}
There are two ways to send advertising out with GAP: The Advertising Data payload and the Scan Response payload. Both payloads are identical and can contain up to 31 bytes of data, but only the advertising data payload is mandatory. The scan response payload is an optional secondary payload that central devices can request.
\subsubsection{Dedicated bonding}


\subsection{GATT - Generic Attribute Profile}
The GATT profile uses the ATT for discovering services, and for reading and writing characteristic values on a peer device. GATT also specifies the format of data contained on the GATT server: it groups ATT attributes into Services and Characteristics, and defines set of queries the GATT Client can use to discover services, characteristics.

\subsubsection{GATT BLE Roles}

\begin{itemize}
\item \emph{GATT Server Role} - The GATT server stores the data and accepts GATT client requests, commands and confirmations. The GATT server sends responses to requests and when configured, sends indication and notifications asynchronously to the GATT client. 
\item \emph{GATT Client Role} - The GATT Client is used to discover services, and their characteristics and descriptors on a peer device. It can also subscribe for notifications or indications that the characteristic on the GATT server has changed its value. 
\end{itemize}

\subsubsection{Attribute Database - GATT-based Profile Hierarchy}
Attributes, as transported by the Attribute Protocol, are formatted as services and characteristics. Services may contain a collection of characteristics. Characteristics contain a single value and any number of descriptors describing the characteristic value. The peripheral device server (ATT server) provides a set of attributes that are stored in a simple lookup database. GATT formats these attributes as services and characteristics. Services may contain a collection of characteristics. Characteristics contain a single value and any number of descriptors describing the characteristic value. A service starts with a service declaration attribute defining its type, i.e. primary or secondary. It is followed by the included services and characteristics. By means of including services, the service can incorporate more complex behavior, and still keep the definition small. A characteristic is assigned to a single value that can be accessed. It is composed of three basic elements: declaration, value and descriptors. The characteristic declaration defines how the data can be accessed. A characteristic descriptor is used to capture the additional properties, e.g., to configure if the characteristic value should be reported upon its change. Together, characteristic declaration and the descriptors define types of action that can be performed on characteristic value.

The security that is required to access a service or a characteristic is defined in ATT database along with service/characteristic declaration. The GATT Server usually does not initiate any security request, but it can. 

\subsection{BNEP -  Bluetooth Network Encapsulation Protocol}
The BNEP protocol is built on top of L2CAP and it is used to transport control and data packets over standard network protocols such as TCP, IPv4 or IPv6.  . BNEP specifies a minimum L2CAP MTU of 1691. 

\subsection{PAN -  Personal Area Networking Profile}
The PAN profile uses BNEP to provide on-demand networking capabilities between Bluetooth devices. The PAN profile defines the following roles:
\begin{itemize}
\item PAN User (PANU)
\item Network Access Point (NAP)
\item Group Ad-hoc Network (GN)
\end{itemize}

PANU is a Bluetooth device that communicates as a client with GN, or NAP, or with another PANU Bluetooth device, through a point-to-point connection. Either the PANU or the other Bluetooth device may terminate the connection at anytime.

NAP is a Bluetooth device that provides the service of routing network packets between PANU by using BNEP and the IP routing mechanism. A NAP can also act as a bridge between Bluetooth networks and other network technologies by using the Ethernet packets.

The GN role enables two or more PANUs to interact with each other through a wireless network without using additional networking hardware. The devices are connected in a piconet where the GN acts as a master and communicates either point-to-point or a point-to-multipoint with a maximum of seven PANU slaves by using BNEP.

Currently, BTstack supports only PANU.

\section{Dual Mode Support}
\subsection{BR/EDR - Basic Rate/Extended Data Rate, or shortly Classic}
\subsection{LE - Low Energy}
Bluetooth Low Energy (BLE) is a Bluetooth technology used for discovering services and optimized for low power consumption - you don't get high data rates, and usually don't keep connection for long periods. The focus is on two different device roles:  devices that provide services and/or can be controlled and devices that consume services and/or control other devices. Devices are first going through the advertising process that is governed by the Generic Access Profile (GAP). Once the connection is established, the communication will be governed by the Generic Attribute Profile (GATT). Both profiles, GAP and GATT, have concepts that describe these two BLE roles:  GAP defines Peripheral and Central, and GATT defines Server and Client role respectively. The GATT roles are not necessarily tied to specific GAP roles and may be specified by higher layer profiles. GATT is built on top of the Attribute Protocol (ATT), which defines how to discover, read, and write attributes on a peer device. In addition, BLE uses two more BT protocols: SMP for for pairing and transport specific key distribution and L2CAP LE variant optimized for connectionless data used by Bluetooth Low Energy devices. 


\subsubsection{Private/random addresses}
To better protect privacy, a LE device can choose use a private i.e. random Bluetooth address. This address changes at a user-specified rate. To allow for later reconnection, the central and peripheral devices exchange their Identity Resolving Keys (IRKs) during bonding. The IRK is used to verify if a new address belongs to a previously bonded device.

\subsubsection{Security manager}
\label{section:security_manager}

\pagebreak 
\section{Examples}
In this section, we will describe a number of examples from the \emph{example/embedded} folder. To allow code-reuse with different platforms as well as with new ports, the low-level initialization of BTstack and the hardware configuration has been extracted to the various \emph{platforms/\$PLATFORM/main.c} files. The examples only contain the platform-independent Bluetooth logic. But let's have a look at the common init code. 

Listing \ref{lst:btstackInit} shows a minimal platform setup for an embedded system with a Bluetooth chipset connected via UART.

\begin{lstlisting}[float, caption=Exemplary platform init in main.c, label=lst:btstackInit]
int main(){}
  // ... hardware init: watchdoch, IOs, timers, etc...

  // setup BTstack memory pools
  btstack_memory_init();

  // select embedded run loop
  run_loop_init(RUN_LOOP_EMBEDDED);
      
  // use logger: format HCI_DUMP_PACKETLOGGER, HCI_DUMP_BLUEZ or HCI_DUMP_STDOUT
  hci_dump_open(NULL, HCI_DUMP_STDOUT);

  // init HCI
  hci_transport_t    * transport = hci_transport_h4_dma_instance();
  remote_device_db_t * remote_db = (remote_device_db_t *) &remote_device_db_memory;
  hci_init(transport, NULL, NULL, remote_db);

  // setup example    
  btstack_main(argc, argv);

  // go
  run_loop_execute();    
\end{lstlisting}

First, BTstack's memory pools are setup up. Then, the standard run loop implementation for embedded systems is selected.

The call to \emph{hci\_dump\_open} configures BTstack to output all Bluetooth packets and it's own debug and error message via printf. The Python script \emph{tools/create\_packet\_log.py} can be used to convert the console output into a Bluetooth PacketLogger format that can be opened by the OS X PacketLogger tool as well as by Wireshark for further inspection. When asking for help, please always include a log created with hci dump of the problem. 

The \emph{hci\_init} call setups HCI to use the HCI H4 Transport implementation. It doesn't provide a special transprot configuration nor a special implementation for a particular Bluetooth chipset. It makes use of the \emph{remote\_device\_db\_memory} implementation that allows for re-connects without a new pairing but doesn't persist the bonding information. 

Finally, it calls \emph{btstack\_main()} of the acutal example before executing the run lopp. 

The examples are grouped like this:
 \input{examples}

% \section{Platforms}

\section{Porting to Other Platforms}

In this chapter, we highlight the BTstack components that need to be adjusted for different hardware platforms.

\subsection{Time Abstraction Layer}
\label{section:timeAbstraction}
BTstack requires a way to learn about passing time. \emph{run\_loop\_embedded.c} supports two different modes: system ticks or a system clock with millisecond resolution. BTstack's timing requirements are quite low as only Bluetooth timeouts in the second range need to be handled.

\subsubsection{Tick Hardware Abstraction}
\label{ssection:tickAbstraction}

If your platform doesn't require a system clock or if you already have a system tick (as it is the default with CMSIS on ARM Cortex devices), you can use that to implement BTstack's time abstraction in \emph{include/btstack/hal\_tick.h>}.

For this, you need to define \emph{HAVE\_TICK} in \emph{btstack-config.h}:
\begin{lstlisting}
#define HAVE_TICK
\end{lstlisting}

Then, you need to implement the functions \emph{hal\_tick\_init} and \emph{hal\_tick\_set\_handler}, which will be called during the initialization of the run loop.
 
\begin{lstlisting}
void hal_tick_init(void);
void hal_tick_set_handler(void (*tick_handler)(void));
int  hal_tick_get_tick_period_in_ms(void);
\end{lstlisting}
 
After BTstack calls \emph{hal\_tick\_init()} and \emph{hal\_tick\_set\_handler(tick\_handler)}, it expects that the \emph{tick\_handler} gets called every \emph{hal\_tick\_get\_tick\_period\_in\_ms()} ms.

\subsubsection{Time MS Hardware Abstraction}
\label{ssection:timeMSAbstraction}
If your platform already has a system clock or it is more convenient to provide such a clock, you can use the Time MS Hardware Abstraction in \emph{include/btstack/hal\_time\_ms.h}.

For this, you need to define \emph{HAVE\_TIME\_MS} in \emph{btstack-config.h}:
\begin{lstlisting}
#define HAVE_TIME_MS
\end{lstlisting}

Then, you need to implement the function \emph{hal\_time\_ms()}, which will be called from BTstack's run loop and when setting a timer for the future. It has to return the time in milliseconds.

\begin{lstlisting}
uint32_t hal_time_ms(void);
\end{lstlisting}

\subsection{Bluetooth Hardware Control API}
\label{section:bt_hw_control}
The Bluetooth hardware control API can provide the HCI layer with a custom initialization script, a vendor-specific baud rate change command, and system power notifications. It is also used to control the power mode of the Bluetooth module, i.e., turning it on/off and putting to sleep. In addition, it provides an error handler \emph{hw\_error} that is called when a Hardware Error is reported by the Bluetooth module. The callback allows for persistent logging or signaling of this failure. \todo{add recipe}

Overall, the struct \emph{bt\_control\_t} encapsulates common functionality that is not covered by the Bluetooth specification. As an example, the \emph{bt\_control\_cc256x\_in-stance} function returns a pointer to a control struct suitable for the CC256x chipset.

 \subsection{HCI Transport Implementation}
 \label{section:hci_transport}
On embedded systems, a Bluetooth module can be connected via USB or an UART port. BTstack implements two UART based protocols for carrying HCI commands, events and data between a host and a Bluetooth module: HCI UART Transport Layer (H4) and H4 with eHCILL support, a lightweight low-power variant by Texas Instruments.

 \subsubsection{HCI UART Transport Layer (H4)}
 \label{section:hciUART}
Most embedded UART interfaces operate on the byte level and generate a processor interrupt when a byte was received. In the interrupt handler, common UART drivers then place the received data in a ring buffer and set a flag for further processing or notify the higher-level code, i.e., in our case the Bluetooth stack.

Bluetooth communication is packet-based and a single packet may contain up to 1021 bytes. Calling a data received handler of the Bluetooth stack for every byte creates an unnecessary overhead. To avoid that, a Bluetooth packet can be read as multiple blocks where the amount of bytes to read is known in advance. Even better would be the use of on-chip DMA modules for these block reads, if available.

%During the BTstack will request up to three reads: first to get the packet type, second to get ACL or Event header, and last one to read the payload. 

The BTstack UART Hardware Abstraction Layer API reflects this design approach and the underlying UART driver has to implement the following API:

 \begin{lstlisting}
void hal_uart_dma_init(void);
void hal_uart_dma_set_block_received(void (*block_handler)(void));
void hal_uart_dma_set_block_sent(void (*block_handler)(void));
int  hal_uart_dma_set_baud(uint32_t baud);
void hal_uart_dma_send_block(const uint8_t *buffer, uint16_t len);
void hal_uart_dma_receive_block(uint8_t *buffer, uint16_t len);
 \end{lstlisting}
 
The main HCI H4 implementations for embedded system is \emph{hci\_h4\_transport-\_dma} function. This function calls the following sequence: \emph{hal\_uart\_dma\_init}, \emph{hal\_uart\_dma\_set\_block\_received} and \emph{hal\_uart\_dma\_set\_block\_sent} functions. \mbox{After} this sequence, the HCI layer will start packet processing by calling \emph{hal\_uart-\_dma\_receive\_block} function. The HAL implementation is responsible for reading the requested amount of bytes, stopping incoming data via the RTS line when the requested amount of data was received and has to call the handler. By this, the HAL implementation can stay generic, while requiring only three callbacks per HCI packet. 

\subsubsection{H4 with eHCILL support}
With the standard H4 protocol interface, it is not possible for either the host nor the baseband controller to enter a sleep mode. Besides the official H5 protocol, various chip vendors came up with proprietary solutions to this. The eHCILL support by Texas Instruments allows both the host and the baseband controller to independently enter sleep mode without loosing their synchronization with the HCI H4 Transport Layer. In addition to the IRQ-driven block-wise RX and TX, eHCILL requires a callback for CTS interrupts.

\begin{lstlisting}
void hal_uart_dma_set_cts_irq_handler(void(*cts_irq_handler)(void));
void hal_uart_dma_set_sleep(uint8_t sleep);
\end{lstlisting}


\subsection{Persistent Storage API}
\label{section:persistent_storage}

On embedded systems there is no generic way to persist data like link keys or remote device names, as every type of a device has its own capabilities, particularities and limitations. The persistent storage API provides an interface to implement concrete drivers for a particular system. As an example and for testing purposes, BTstack provides the memory-only implementation \emph{remote\_device\_db\_memory}. An implementation has to conform to the interface in Listing \ref{persistentDB}.
\\

\begin{lstlisting}[float, caption=Persistent Storage Interface., label=persistentDB]
typedef struct {
    // management
    void (*open)();
    void (*close)();
    
    // link key
    int  (*get_link_key)(bd_addr_t bd_addr, link_key_t link_key);
    void (*put_link_key)(bd_addr_t bd_addr, link_key_t key);
    void (*delete_link_key)(bd_addr_t bd_addr);
    
    // remote name
    int (*get_name)(bd_addr_t bd_addr, device_name_t *device_name);
    void(*put_name)(bd_addr_t bd_addr, device_name_t *device_name);
    void(*delete_name)(bd_addr_t bd_addr);
} remote_device_db_t;

\end{lstlisting}

\section{Integrating with Existing Systems}

While the run loop provided by BTstack is sufficient for new designs, BTstack is often used with or added to existing projects. In this case, the run loop, data sources, and timers may need to be adapted. The following two sections provides a guideline for single and multi-threaded environments.

To simplify the discussion, we'll consider an application split into "Main \mbox{Application}", "Communication Logic", and "BTstack". The Communication Logic contains the packet handler (PH) that handles all asynchronous events and data packets from BTstack. The Main Application makes use of the Communication Logic for its Bluetooth communication.

\subsection{Adapting BTstack for Single-Threaded Environments}
\label{section:singlethreading}

In a single-threaded environment, all application components run on the same (single) thread and use direct function calls as shown in Figure \ref{fig:BTstackSingle}.

\begin{figure}[htbp] %  figure placement: here, top, bottom, or page
   \centering
   \includegraphics[width=0.3\textwidth]{picts/singlethreading-btstack.pdf} 
   \caption{BTstack in single-threaded environment. }
   \label{fig:BTstackSingle}
\end{figure}

BTstack provides a basic run loop that supports the concept of data sources and timers, which can be registered centrally. This works well when working with a small MCU and without an operating system.
To adapt to a basic operating system or a different scheduler, BTstack's run loop can be implemented based on the functions and mechanism of the existing system. 

Currently, we have two examples for this:
\begin{itemize}
\item \emph{run\_loop\_cocoa.c} is an implementation for the CoreFoundation Framework used in OS X and iOS. All run loop functions are implemented in terms of CoreFoundation calls, data sources and timers are modeled as CFSockets and CFRunLoopTimer respectively. 
\item \emph{run\_loop\_posix.c} is an implementation for POSIX compliant systems. The data sources are modeled as file descriptors and managed in a linked list. Then, the\emph{select} function is used to wait for the next file descriptor to become ready or timer to expire. 
\end{itemize}

\subsection{Adapting BTstack for Multi-Threaded Environments}
\label{section:multithreading}

The basic execution model of BTstack is a general while loop. Aside from interrupt-driven UART and timers, everything happens in sequence. 
When using BTstack in a multi-threaded environment,  this assumption has to stay valid - at least with respect to BTstack. For this, there are two common options:

\begin{figure}[ht]
\begin{minipage}[b]{\linewidth}
\centering
   \includegraphics[width=0.8\textwidth]{picts/multithreading-monolithic.pdf} 
   \caption{BTstack in multi-threaded environment - monolithic solution.}
   \label{fig:BTstackMonolithic}
\vspace{0.8cm}
\end{minipage}

\begin{minipage}[b]{\linewidth}
\centering
   \includegraphics[width=0.8\textwidth]{picts/multithreading-btdaemon.pdf} 
   \caption{BTstack in multi-threaded environment - solution with daemon.}
   \label{fig:BTstackDaemon}
\end{minipage}
\end{figure}

\begin{itemize}
\item The Communication Logic is implemented on a dedicated BTstack thread, and the Main Application communicates with the BTstack thread via application-specific messages over an Interprocess Communication (IPC) as depicted in Figure \ref{fig:BTstackMonolithic}. This option results in less code and quick adaption.
\item BTstack must be extended to run standalone, i.e, as a Daemon, on a dedicated thread and the Main Application controls this daemon via BTstack extended HCI command over IPC - this is used for the non-embedded version of BTstack e.g., on the iPhone and it is depicted in Figure \ref{fig:BTstackDaemon}. This option requires more code but provides more flexibility.
\end{itemize}


\pagebreak

\appendix
	% !TEX root = btstack_gettingstarted.tex

% !TEX root = btstack_gettingstarted.tex
\section{Run Loop}
\label{appendix:api_run_loop}
$ $
\begin{lstlisting}
// Set timer based on current time in milliseconds.
void run_loop_set_timer(timer_source_t *a, uint32_t timeout_in_ms);

// Set callback that will be executed when timer expires.
void run_loop_set_timer_handler(timer_source_t *ts, void (*process)(timer_source_t *_ts));

// Add/Remove timer source.
void run_loop_add_timer(timer_source_t *timer); 
int  run_loop_remove_timer(timer_source_t *timer);

// Init must be called before any other run_loop call. 
// Use RUN_LOOP_EMBEDDED for embedded devices.
void run_loop_init(RUN_LOOP_TYPE type);

// Set data source callback.
void run_loop_set_data_source_handler(data_source_t *ds, int (*process)(data_source_t *_ds));

// Add/Remove data source.
void run_loop_add_data_source(data_source_t *dataSource);
int  run_loop_remove_data_source(data_source_t *dataSource);

// Execute configured run loop. This function does not return.
void run_loop_execute(void);

// hack to fix HCI timer handling
#ifdef HAVE_TICK
// Sets how many miliseconds has one tick.
uint32_t embedded_ticks_for_ms(uint32_t time_in_ms);
// Queries the current time in ticks.
uint32_t embedded_get_ticks(void);
// Queries the current time in ms
uint32_t embedded_get_time_ms(void);
// Allows to update BTstack system ticks based on another already 
// existing clock.
void embedded_set_ticks(uint32_t ticks);
#endif
#ifdef EMBEDDED
// Sets an internal flag that is checked in the critical section
// just before entering sleep mode. Has to be called by the interupt
// handler of a data source to signal the run loop that a new data 
// is available.
void embedded_trigger(void);    
// Execute run_loop once. It can be used to integrate BTstack's 
// timer and data source processing into a foreign run runloop 
// (it is not recommended).
void embedded_execute_once(void);
#endif
\end{lstlisting}
\pagebreak

% !TEX root = btstack_gettingstarted.tex
\section{Host Controller Interface (HCI)}
\label{appendix:api_hci}
$ $
\begin{lstlisting}
le_connection_parameter_range_t gap_le_get_connection_parameter_range();
void gap_le_set_connection_parameter_range(le_connection_parameter_range_t range);

/* LE Client Start */

le_command_status_t le_central_start_scan(void);
le_command_status_t le_central_stop_scan(void);
le_command_status_t le_central_connect(bd_addr_t addr, bd_addr_type_t addr_type);
le_command_status_t le_central_connect_cancel(void);
le_command_status_t gap_disconnect(hci_con_handle_t handle);
void le_central_set_scan_parameters(uint8_t scan_type, uint16_t scan_interval, uint16_t scan_window);

/* LE Client End */
    
void hci_connectable_control(uint8_t enable);
void hci_close(void);

// new functions replacing: 
// hci_can_send_packet_now[_using_packet_buffer]
int hci_can_send_command_packet_now(void);
    
// gets local address 
void hci_local_bd_addr(bd_addr_t address_buffer);

// Set up HCI. Needs to be called before any other function.
void hci_init(hci_transport_t *transport, void *config, bt_control_t *control, remote_device_db_t const* remote_device_db);

// Set class of device that will be set during Bluetooth init.
void hci_set_class_of_device(uint32_t class_of_device);

// Set Public BD ADDR - passed on to Bluetooth chipset if supported 
// in bt_control_h
void hci_set_bd_addr(bd_addr_t addr);

// Registers a packet handler. Used if L2CAP is not used (rarely). 
void hci_register_packet_handler(void (*handler)(uint8_t packet_type, uint8_t *packet, uint16_t size));

// Requests the change of BTstack power mode.
int  hci_power_control(HCI_POWER_MODE mode);

// Allows to control if device is discoverable. OFF by default.
void hci_discoverable_control(uint8_t enable);

// Creates and sends HCI command packets based on a template and 
// a list of parameters. Will return error if outgoing data buffer 
// is occupied. 
int hci_send_cmd(const hci_cmd_t *cmd, ...);

// Deletes link key for remote device with baseband address.
void hci_drop_link_key_for_bd_addr(bd_addr_t addr);

/* Configure Secure Simple Pairing */

// Enable will enable SSP during init.
void hci_ssp_set_enable(int enable);

// If set, BTstack will respond to io capability request using 
// authentication requirement.
void hci_ssp_set_io_capability(int ssp_io_capability);
void hci_ssp_set_authentication_requirement(int authentication_requirement);

// If set, BTstack will confirm a numberic comparion and enter
// '000000' if requested.
void hci_ssp_set_auto_accept(int auto_accept);

// Get addr type and address used in advertisement packets.
void hci_le_advertisement_address(uint8_t * addr_type, bd_addr_t addr);
\end{lstlisting}
\pagebreak

% !TEX root = btstack_gettingstarted.tex
\section{L2CAP API}
\label{appendix:api_l2cap}
$ $
\begin{lstlisting}
// Set up L2CAP and register L2CAP with HCI layer.
void l2cap_init(void);

// Registers a packet handler that handles HCI and general BTstack
// events.
void l2cap_register_packet_handler(void (*handler)(void * connection, uint8_t packet_type, uint16_t channel, uint8_t *packet, uint16_t size));

// Creates L2CAP channel to the PSM of a remote device with baseband
// address. A new baseband connection will be initiated if needed.
void l2cap_create_channel_internal(void * connection, btstack_packet_handler_t packet_handler, bd_addr_t address, uint16_t psm, uint16_t mtu);

// Disconencts L2CAP channel with given identifier. 
void l2cap_disconnect_internal(uint16_t local_cid, uint8_t reason);

// Queries the maximal transfer unit (MTU) for L2CAP channel with
// given identifier. 
uint16_t l2cap_get_remote_mtu_for_local_cid(uint16_t local_cid);

// Sends L2CAP data packet to the channel with given identifier.
int l2cap_send_internal(uint16_t local_cid, uint8_t *data, uint16_t len);

// Registers L2CAP service with given PSM and MTU, and assigns a
// packet handler. On embedded systems, use NULL for connection
// parameter.
void l2cap_register_service_internal(void *connection, btstack_packet_handler_t packet_handler, uint16_t psm, uint16_t mtu);

// Unregisters L2CAP service with given PSM.  On embedded systems,
// use NULL for connection parameter.
void l2cap_unregister_service_internal(void *connection, uint16_t psm);

// Accepts/Deny incoming L2CAP connection.
void l2cap_accept_connection_internal(uint16_t local_cid);
void l2cap_decline_connection_internal(uint16_t local_cid, uint8_t reason);
\end{lstlisting}
\pagebreak
% !TEX root = btstack_gettingstarted.tex
\section{RFCOMM}
\label{appendix:api_rfcomm}
$ $
\begin{lstlisting}
// Set up RFCOMM.
void rfcomm_init(void);

// Set security level required for incoming connections, need to be 
// called before registering services.
void rfcomm_set_required_security_level(gap_security_level_t security_level);

// Register packet handler.
void rfcomm_register_packet_handler(void (*handler)(void * connection, uint8_t packet_type, uint16_t channel, uint8_t *packet, uint16_t size));

// Creates RFCOMM connection (channel) to a given server channel on 
// a remote device with baseband address. A new baseband connection 
// will be initiated if necessary. This channel will automatically 
// provide enough credits to the remote side
void rfcomm_create_channel_internal(void * connection, bd_addr_t addr, uint8_t channel);

// Creates RFCOMM connection (channel) to a given server channel on 
// a remote device with baseband address. new baseband connection 
// will be initiated if necessary. This channel will use explicit 
// credit management. During channel establishment, an initial  
// amount of credits is provided.
void rfcomm_create_channel_with_initial_credits_internal(void * connection, bd_addr_t addr, uint8_t server_channel, uint8_t initial_credits);

// Disconencts RFCOMM channel with given identifier. 
void rfcomm_disconnect_internal(uint16_t rfcomm_cid);

// Registers RFCOMM service for a server channel and a maximum frame 
// size, and assigns a packet handler. On embedded systems, use NULL 
// for connection parameter. This channel provides automatically  
// enough credits to the remote side.
void rfcomm_register_service_internal(void * connection, uint8_t channel, uint16_t max_frame_size);

// Registers RFCOMM service for a server channel and a maximum frame 
// size, and assigns a packet handler. On embedded systems, use NULL 
// for connection parameter. This channel will use explicit credit
// management. During channel establishment, an initial amount of  
// credits is provided.
void rfcomm_register_service_with_initial_credits_internal(void * connection, uint8_t channel, uint16_t max_frame_size, uint8_t initial_credits);

// Unregister RFCOMM service.
void rfcomm_unregister_service_internal(uint8_t service_channel);

// Accepts/Deny incoming RFCOMM connection.
void rfcomm_accept_connection_internal(uint16_t rfcomm_cid);
void rfcomm_decline_connection_internal(uint16_t rfcomm_cid);

// Grant more incoming credits to the remote side for the given 
// RFCOMM channel identifier.
void rfcomm_grant_credits(uint16_t rfcomm_cid, uint8_t credits);

// Checks if RFCOMM can send packet. Returns yes if packet can be 
// sent.
int rfcomm_can_send_packet_now(uint16_t rfcomm_cid);

// Sends RFCOMM data packet to the RFCOMM channel with given 
// identifier.
int  rfcomm_send_internal(uint16_t rfcomm_cid, uint8_t *data, uint16_t len);

// Sends Local Lnie Status, see LINE_STATUS_..
int rfcomm_send_local_line_status(uint16_t rfcomm_cid, uint8_t line_status);

// Sned local modem status. see MODEM_STAUS_..
int rfcomm_send_modem_status(uint16_t rfcomm_cid, uint8_t modem_status);

// Configure remote port 
int rfcomm_send_port_configuration(uint16_t rfcomm_cid, rpn_baud_t baud_rate, rpn_data_bits_t data_bits, rpn_stop_bits_t stop_bits, rpn_parity_t parity, rpn_flow_control_t flow_control);

// Query remote port 
int rfcomm_query_port_configuration(uint16_t rfcomm_cid);

// Allow to create rfcomm packet in outgoing buffer.
int       rfcomm_reserve_packet_buffer(void);
void      rfcomm_release_packet_buffer(void);
uint8_t * rfcomm_get_outgoing_buffer(void);
uint16_t  rfcomm_get_max_frame_size(uint16_t rfcomm_cid);
int       rfcomm_send_prepared(uint16_t rfcomm_cid, uint16_t len);
\end{lstlisting}
\pagebreak

% !TEX root = btstack_gettingstarted.tex
\section{SDP API}
\label{appendix:api_sdp}
$ $
\begin{lstlisting}
// Set up SDP.
void sdp_init(void);

// Register service record internally - this version does not copy
// the record therefore it must be forever accessible.
// Preconditions:
//     - AttributeIDs are in ascending order;
//     - ServiceRecordHandle is first attribute and valid.
// @returns ServiceRecordHandle or 0 if registration failed.
uint32_t sdp_register_service_internal(void *connection, service_record_item_t * record_item);

// Unregister service record internally.
void sdp_unregister_service_internal(void *connection, uint32_t service_record_handle);
\end{lstlisting}
\pagebreak
% !TEX root = btstack_gettingstarted.tex
\section{SDP Client API}
\label{appendix:api_sdp_client}
$ $
\begin{lstlisting}
/* SDP Client */

// Queries the SDP service of the remote device given a service 
// search pattern and a list of attribute IDs. The remote data is 
// handled by the SDP parser. The SDP parser delivers attribute 
// values and done event via a registered callback.
void sdp_client_query(bd_addr_t remote, uint8_t * des_serviceSearchPattern, uint8_t * des_attributeIDList);



/* SDP Parser */

// Basic SDP Parser event type.
typedef enum sdp_parser_event_type {
    SDP_PARSER_ATTRIBUTE_VALUE = 1,
    SDP_PARSER_COMPLETE,
} sdp_parser_event_type_t;

typedef struct sdp_parser_event {
    uint8_t type;
} sdp_parser_event_t;

// SDP Parser event to deliver an attribute value byte by byte.
typedef struct sdp_parser_attribute_value_event {
    uint8_t type;
    int record_id;
    uint16_t attribute_id;
    uint32_t attribute_length;
    int data_offset;
    uint8_t data;
} sdp_parser_attribute_value_event_t;

// SDP Parser event to indicate that parsing is complete.
typedef struct sdp_parser_complete_event {
    uint8_t type;
    uint8_t status; // 0 == OK
} sdp_parser_complete_event_t;

// Registers a callback to receive attribute value data and parse  
// complete event. 
void sdp_parser_register_callback(void (*sdp_callback)(sdp_parser_event_t* event));
\end{lstlisting}
\pagebreak
% !TEX root = btstack_gettingstarted.tex
\section{SDP RFCOMM Query API}
\label{appendix:api_sdp_queries}
$ $
\begin{lstlisting}
/** 
 * @brief SDP Query RFCOMM event to deliver channel number and service name byte by byte.
 */
typedef struct sdp_query_rfcomm_service_event {
    uint8_t type;
    uint8_t channel_nr;
    uint8_t * service_name;
} sdp_query_rfcomm_service_event_t;

/** 
 * @brief Registers a callback to receive RFCOMM service and query complete event. 
 */
void sdp_query_rfcomm_register_callback(void(*sdp_app_callback)(sdp_query_event_t * event, void * context), void * context);

void sdp_query_rfcomm_deregister_callback();

/** 
 * @brief Searches SDP records on a remote device for RFCOMM services with a given UUID.
 */
void sdp_query_rfcomm_channel_and_name_for_uuid(bd_addr_t remote, uint16_t uuid);

/** 
 * @brief Searches SDP records on a remote device for RFCOMM services with a given service search pattern.
 */
void sdp_query_rfcomm_channel_and_name_for_search_pattern(bd_addr_t remote, uint8_t * des_serviceSearchPattern);
\end{lstlisting}
\pagebreak

% !TEX root = btstack_gettingstarted.tex
\section{GATT Client API}
\label{appendix:api_gatt_client}
$ $
\begin{lstlisting}
// Set up GATT client.
void gatt_client_init();

// Register callback (packet handler) for gatt client. Returns gatt client ID.
uint16_t gatt_client_register_packet_handler (gatt_client_callback_t callback);

// Unregister callback (packet handler) for gatt client.
void gatt_client_unregister_packet_handler(uint16_t gatt_client_id);

// Returns the GATT client context for the specified handle.
// gatt_client_t * get_gatt_client_context_for_handle(uint16_t con_handle);

// Returns if the gatt client is ready to receive a query. It is used with daemon.
int gatt_client_is_ready(uint16_t handle);

// Discovers all primary services. For each found service, an
// le_service_event_t with type set to GATT_SERVICE_QUERY_RESULT
// will be generated and passed to the registered callback.
// The gatt_complete_event_t, with type set to GATT_QUERY_COMPLETE,
// marks the end of discovery.
le_command_status_t gatt_client_discover_primary_services(uint16_t gatt_client_id, uint16_t con_handle);

// Discovers a specific primary service given its UUID. This service
// may exist multiple times. For each found service, an
// le_service_event_t with type set to GATT_SERVICE_QUERY_RESULT
// will be generated and passed to the registered callback.
// The gatt_complete_event_t, with type set to GATT_QUERY_COMPLETE,
// marks the end of discovery.
le_command_status_t gatt_client_discover_primary_services_by_uuid16(uint16_t gatt_client_id, uint16_t con_handle, uint16_t uuid16);
le_command_status_t gatt_client_discover_primary_services_by_uuid128(uint16_t gatt_client_id, uint16_t con_handle, const uint8_t * uuid);

// Finds included services within the specified service. For each
// found included service, an le_service_event_t with type set to
// GATT_INCLUDED_SERVICE_QUERY_RESULT will be generated and passed
// to the registered callback. The gatt_complete_event_t with type
// set to GATT_QUERY_COMPLETE, marks the end of discovery.
//
// Information about included service type (primary/secondary) can
// be retrieved either by sending an ATT find information request
// for the returned start group handle (returning the handle and
// the UUID for primary or secondary service) or by comparing the
// service to the list of all primary services.
le_command_status_t gatt_client_find_included_services_for_service(uint16_t gatt_client_id, uint16_t con_handle, le_service_t *service);

// Discovers all characteristics within the specified service. For
// each found characteristic, an le_characteristics_event_t with 
// type set to GATT_CHARACTERISTIC_QUERY_RESULT will be generated 
// and passed to the registered callback. The gatt_complete_event_t 
// with type set to GATT_QUERY_COMPLETE, marks the end of discovery.
le_command_status_t gatt_client_discover_characteristics_for_service(uint16_t gatt_client_id, uint16_t con_handle, le_service_t *service);

// The following four functions are used to discover all 
// characteristics within the specified service or handle range, and 
// return those that match the given UUID. For each found
// characteristic, an le_characteristic_event_t with type set to   
// GATT_CHARACTERISTIC_QUERY_RESULT will be generated and passed to 
// the registered callback. The gatt_complete_event_t with type set
// to GATT_QUERY_COMPLETE, marks the end of discovery.
le_command_status_t gatt_client_discover_characteristics_for_handle_range_by_uuid16(uint16_t gatt_client_id, uint16_t con_handle, uint16_t start_handle, uint16_t end_handle, uint16_t uuid16);
le_command_status_t gatt_client_discover_characteristics_for_handle_range_by_uuid128(uint16_t gatt_client_id, uint16_t con_handle, uint16_t start_handle, uint16_t end_handle, uint8_t * uuid);
le_command_status_t gatt_client_discover_characteristics_for_service_by_uuid16 (uint16_t gatt_client_id, uint16_t con_handle, le_service_t *service, uint16_t  uuid16);
le_command_status_t gatt_client_discover_characteristics_for_service_by_uuid128(uint16_t gatt_client_id, uint16_t con_handle, le_service_t *service, uint8_t * uuid128);

// Discovers attribute handle and UUID of a characteristic 
// descriptor within the specified characteristic. For each found
// descriptor, an le_characteristic_descriptor_event_t with 
// type set to GATT_CHARACTERISTIC_DESCRIPTOR_QUERY_RESULT will be  
// generated and passed to the registered callback. The  
// gatt_complete_event_t with type set to GATT_QUERY_COMPLETE, marks 
// the end of discovery.
le_command_status_t gatt_client_discover_characteristic_descriptors(uint16_t gatt_client_id, uint16_t con_handle, le_characteristic_t *characteristic);

// Reads the characteristic value using the characteristic's value
// handle. If the characteristic value is found, an 
// le_characteristic_value_event_t with type set to 
// GATT_CHARACTERISTIC_VALUE_QUERY_RESULT will be generated and
// passed to the registered callback. The gatt_complete_event_t 
// with type set to GATT_QUERY_COMPLETE, marks the end of read.
le_command_status_t gatt_client_read_value_of_characteristic(uint16_t gatt_client_id, uint16_t con_handle, le_characteristic_t *characteristic);
le_command_status_t gatt_client_read_value_of_characteristic_using_value_handle(uint16_t gatt_client_id, uint16_t con_handle, uint16_t characteristic_value_handle);

// Reads the long characteristic value using the characteristic's 
// value handle. The value will be returned in several blobs. For 
// each blob, an le_characteristic_value_event_t with type set to 
// GATT_CHARACTERISTIC_VALUE_QUERY_RESULT and updated value offset
// will be generated and passed to the registered callback. The 
// gatt_complete_event_t with type set to GATT_QUERY_COMPLETE, marks
// the end of read.
le_command_status_t gatt_client_read_long_value_of_characteristic(uint16_t gatt_client_id, uint16_t con_handle, le_characteristic_t *characteristic);
le_command_status_t gatt_client_read_long_value_of_characteristic_using_value_handle(uint16_t gatt_client_id, uint16_t con_handle, uint16_t characteristic_value_handle);
    
// Writes the characteristic value using the characteristic's value
// handle without an acknowledgement that the write was successfully
// performed.
le_command_status_t gatt_client_write_value_of_characteristic_without_response(uint16_t gatt_client_id, uint16_t con_handle, uint16_t characteristic_value_handle, uint16_t length, uint8_t * data);

// Writes the authenticated characteristic value using the
// characteristic's value handle without an acknowledgement
// that the write was successfully performed.
le_command_status_t gatt_client_signed_write_without_response(uint16_t gatt_client_id, uint16_t con_handle, uint16_t handle, uint16_t message_len, uint8_t * message, sm_key_t csrk, uint32_t sgn_counter);

// Writes the characteristic value using the characteristic's value
// handle. The gatt_complete_event_t with type set to
// GATT_QUERY_COMPLETE, marks the end of write. The write is
// successfully performed, if the event's status field is set to 0.
le_command_status_t gatt_client_write_value_of_characteristic(uint16_t gatt_client_id, uint16_t con_handle, uint16_t characteristic_value_handle, uint16_t length, uint8_t * data);
le_command_status_t gatt_client_write_long_value_of_characteristic(uint16_t gatt_client_id, uint16_t con_handle, uint16_t characteristic_value_handle, uint16_t length, uint8_t * data);

// Writes of the long characteristic value using the
// characteristic's value handle. It uses server response to
// validate that the write was correctly received.
// The gatt_complete_event_t with type set to GATT_QUERY_COMPLETE
// marks the end of write. The write is successfully performed, if
// the event's status field is set to 0.
le_command_status_t gatt_client_reliable_write_long_value_of_characteristic(uint16_t gatt_client_id, uint16_t con_handle, uint16_t characteristic_value_handle, uint16_t length, uint8_t * data);

// Reads the characteristic descriptor using its handle. If the
// characteristic descriptor is found, an
// le_characteristic_descriptor_event_t with type set to
// GATT_CHARACTERISTIC_DESCRIPTOR_QUERY_RESULT will be generated
// and passed to the registered callback. The gatt_complete_event_t
// with type set to GATT_QUERY_COMPLETE, marks the end of read.
le_command_status_t gatt_client_read_characteristic_descriptor(uint16_t gatt_client_id, uint16_t con_handle, le_characteristic_descriptor_t * descriptor);

// Reads the long characteristic descriptor using its handle. It
// will be returned in several blobs.  For each blob, an
// le_characteristic_descriptor_event_t with type set to
// GATT_CHARACTERISTIC_DESCRIPTOR_QUERY_RESULT will be generated
// and passed to the registered callback. The gatt_complete_event_t
// with type set to GATT_QUERY_COMPLETE, marks the end of read.
le_command_status_t gatt_client_read_long_characteristic_descriptor(uint16_t gatt_client_id, uint16_t con_handle, le_characteristic_descriptor_t * descriptor);

// Writes the characteristic descriptor using its handle.
// The gatt_complete_event_t with type set to
// GATT_QUERY_COMPLETE, marks the end of write. The write is
// successfully performed, if the event's status field is set to 0.
le_command_status_t gatt_client_write_characteristic_descriptor(uint16_t gatt_client_id, uint16_t con_handle, le_characteristic_descriptor_t * descriptor, uint16_t length, uint8_t * data);
le_command_status_t gatt_client_write_long_characteristic_descriptor(uint16_t gatt_client_id, uint16_t con_handle, le_characteristic_descriptor_t * descriptor, uint16_t length, uint8_t * data);

// Writes the client characteristic configuration of the specified 
// characteristic. It is used to subscribe for notifications or 
// indications of the characteristic value. For notifications or 
// indications specify:
// GATT_CLIENT_CHARACTERISTICS_CONFIGURATION_NOTIFICATION resp. 
// GATT_CLIENT_CHARACTERISTICS_CONFIGURATION_INDICATION 
// as configuration value.
le_command_status_t gatt_client_write_client_characteristic_configuration(uint16_t gatt_client_id, uint16_t con_handle, le_characteristic_t * characteristic, uint16_t configuration);
\end{lstlisting}
\pagebreak
    
	% !TEX root = btstack_gettingstarted.tex
\section{Events and Errors}
\label{appendix:api_events_and_errors}

L2CAP events and data packets are delivered to the packet handler specified by \emph{l2cap\_register\_service} resp. \emph{l2cap\_create\_channel\_internal}. Data packets have the L2CAP\_DATA\_PACKET packet type. L2CAP provides the following events:
\begin{itemize}
\item L2CAP\_EVENT\_CHANNEL\_OPENED - sent if channel establishment is done. Status not equal zero indicates an error. Possible errors: out of memory; connection terminated by local host, when the connection to remote device fails.
\item L2CAP\_EVENT\_CHANNEL\_CLOSED - emitted when channel is closed. No status information is provided.
\item L2CAP\_EVENT\_INCOMING\_CONNECTION - received when the connection is requested  by remote. Connection accept and decline are performed with \emph{l2cap\_accept\_connection\_internal} and \emph{l2cap\_decline\_connecti-on\_internal} respectively.
\item L2CAP\_EVENT\_CREDITS - emitted when there is a chance to send a new L2CAP packet. BTstack does not buffer packets. Instead, it requires the application to retry sending if BTstack cannot deliver a packet to the Bluetooth module. In this case, the l2cap\_send\_internal will return an error. 
\item L2CAP\_EVENT\_SERVICE\_REGISTERED - Status not equal zero indicates an error. Possible errors: service is already registered; MAX\_NO-\_L2CAP\_SERVICES (defined in config.h) already registered.
\end{itemize}
 
\begin{table*}[bp]
\caption{L2CAP Events}
\begin{tabular}{p{1cm}p{10cm}c}\toprule
\multicolumn{2}{c}{Event / Event Parameters (size in bits) } & Event Code\\ 
\midrule
\multicolumn{2}{l}{L2CAP\_EVENT\_CHANNEL\_OPENED} & 0x70\\
&\emph{event(8), len(8), status(8), address(48), handle(16)}\\
&\emph{psm(16), local\_cid(16), remote\_cid(16), local\_mtu(16), }\\
&\emph{remote\_mtu(16)}\\
\\
\multicolumn{2}{l}{L2CAP\_EVENT\_CHANNEL\_CLOSED} & 0x71\\
&\emph{event (8), len(8), channel(16)}\\
\\
\multicolumn{2}{l}{L2CAP\_EVENT\_INCOMING\_CONNECTION}& 0x72\\
&\emph{event(8), len(8), address(48), handle(16), psm (16), }\\
&\emph{local\_cid(16), remote\_cid (16)}\\
\\
\multicolumn{2}{l}{L2CAP\_EVENT\_CREDITS} & 0x74\\
&\emph{event(8), len(8), local\_cid(16), credits(8)}\\
\\
\multicolumn{2}{l}{L2CAP\_EVENT\_SERVICE\_REGISTERED} & 0x75\\
&\emph{event(8), len(8), status(8), psm(16)}\\
\\
\bottomrule
 \label{table:l2capEvents}
\end{tabular}
\end{table*}

\pagebreak

All RFCOMM events and data packets are currently delivered to the packet handler specified by \emph{rfcomm\_register\_packet\_handler}. Data packets have the \mbox{RFCOMM}\_DATA\_PACKET packet type. Here is the list of events provided by RFCOMM:
\begin{itemize}
\item RFCOMM\_EVENT\_INCOMING\_CONNECTION - received when the connection is requested by remote. Connection accept and decline are performed with \emph{rfcomm\_accept\_connection\_internal} and \emph{rfcomm\_decline\_con-nection\_internal} respectively.
\item RFCOMM\_EVENT\_CHANNEL\_CLOSED - emitted when channel is closed. No status information is provided.
\item RFCOMM\_EVENT\_OPEN\_CHANNEL\_COMPLETE - sent if channel establishment is done. Status not equal zero indicates an error. Possible errors: an L2CAP error, out of memory.
\item RFCOMM\_EVENT\_CREDITS - The application can resume sending when this even is received. See Section \ref{section:flowcontrol} for more on RFCOMM credit-based flow-control.
\item RFCOMM\_EVENT\_SERVICE\_REGISTERED - Status not equal zero indicates an error. Possible errors: service is already registered; MAX-\_NO\_RFCOMM\_SERVICES (defined in config.h) already registered.
\end{itemize}

\begin{table*}[bp]
\caption{RFCOMM Events}
\begin{tabular}{p{1cm}p{10cm}c}\toprule
\multicolumn{2}{c}{Event / Event Parameters (size in bits)} & Event Code\\ 
\midrule

\multicolumn{2}{l}{RFCOMM\_EVENT\_OPEN\_CHANNEL\_COMPLETE} & 0x80\\
&\emph{event(8), len(8), status(8), address(48), handle(16), }\\
&\emph{server\_channel(8), rfcomm\_cid(16), max\_frame\_size(16)}\\
\\
\multicolumn{2}{l}{RFCOMM\_EVENT\_CHANNEL\_CLOSED} & 0x81\\
&\emph{event(8), len(8), rfcomm\_cid(16)}\\
\\
\multicolumn{2}{l}{RFCOMM\_EVENT\_INCOMING\_CONNECTION}& 0x82\\
&\emph{event(8), len(8), address(48), channel (8),} \\
& \emph{rfcomm\_cid(16)}\\
\\
\multicolumn{2}{l}{RFCOMM\_EVENT\_CREDITS} & 0x84\\
&\emph{event(8), len(8), rfcomm\_cid(16), credits(8)}\\
\\
\multicolumn{2}{l}{RFCOMM\_EVENT\_SERVICE\_REGISTERED} & 0x85\\
&\emph{event(8), len(8), status(8), rfcomm server channel\_id(8)}\\
\bottomrule
 \label{table:rfcommEvents}
\end{tabular}
\end{table*}


\begin{table}
\caption{Errors}
\begin{tabular}{lc}\toprule
Error & Error Code\\ 
\midrule
{\tiny BTSTACK\_MEMORY\_ALLOC\_FAILED} & {\tiny 0x56}\\
{\tiny BTSTACK\_ACL\_BUFFERS\_FULL} & {\tiny 0x57}\\
{\tiny L2CAP\_COMMAND\_REJECT\_REASON\_COMMAND\_NOT\_UNDERSTOOD} & {\tiny 0x60}\\
{\tiny L2CAP\_COMMAND\_REJECT\_REASON\_SIGNALING\_MTU\_EXCEEDED} & {\tiny 0x61}\\
{\tiny L2CAP\_COMMAND\_REJECT\_REASON\_INVALID\_CID\_IN\_REQUEST} & {\tiny 0x62}\\
{\tiny L2CAP\_CONNECTION\_RESPONSE\_RESULT\_SUCCESSFUL} & {\tiny 0x63}\\
{\tiny L2CAP\_CONNECTION\_RESPONSE\_RESULT\_PENDING} & {\tiny 0x64}\\
{\tiny L2CAP\_CONNECTION\_RESPONSE\_RESULT\_REFUSED\_PSM} & {\tiny 0x65}\\
{\tiny L2CAP\_CONNECTION\_RESPONSE\_RESULT\_REFUSED\_SECURITY} & {\tiny 0x66}\\
{\tiny L2CAP\_CONNECTION\_RESPONSE\_RESULT\_REFUSED\_RESOURCES} & {\tiny 0x65}\\
{\tiny L2CAP\_CONFIG\_RESPONSE\_RESULT\_SUCCESSFUL} & {\tiny 0x66}\\
{\tiny L2CAP\_CONFIG\_RESPONSE\_RESULT\_UNACCEPTABLE\_PARAMS} & {\tiny 0x67}\\
{\tiny L2CAP\_CONFIG\_RESPONSE\_RESULT\_REJECTED} & {\tiny 0x68}\\
{\tiny L2CAP\_CONFIG\_RESPONSE\_RESULT\_UNKNOWN\_OPTIONS} & {\tiny 0x69}\\
{\tiny L2CAP\_SERVICE\_ALREADY\_REGISTERED} & {\tiny 0x6a}\\
{\tiny RFCOMM\_MULTIPLEXER\_STOPPED} & {\tiny 0x70}\\
{\tiny RFCOMM\_CHANNEL\_ALREADY\_REGISTERED} & {\tiny 0x71}\\
{\tiny RFCOMM\_NO\_OUTGOING\_CREDITS} & {\tiny 0x72}\\
{\tiny SDP\_HANDLE\_ALREADY\_REGISTERED} & {\tiny 0x80}\\
\bottomrule
\label{table:sdpErrors}
\end{tabular}
\end{table}

\begin{minipage}[t][5cm][t]{\textwidth}
\end{minipage}
\pagebreak

\pagebreak

\section{Revision History}
\label{appendix:revision_history}

\begin{table}[!htbp]
\begin{tabular*}{\textwidth}{lp{3.5cm}p{8.5cm}}\toprule
Rev & Date & Comments\\ 
\midrule
1.x & April 3, 2015 & Added Time MS Hardware Abstraction.\\
1.3 & November 6, 2014 & Introducing GATT client and server. Work in progress.\\
1.2 & November 1, 2013 & Explained Secure Simple Pairing in "Pairing of devices".\\
1.1 & August 30, 2013 & Introduced SDP client. Updated Quick Recipe on "Query remote SDP service".\\
\bottomrule
\end{tabular*}
\end{table}

%\section {TODO}
%
%\section {Maybe Missing}
%\begin{itemize}
%\item profiles in BTstack (SPP)
%\item More examples: HID mouse host, HID mouse device, HID keyboard device, BLE Client, ..
%\end{itemize}
%


% \bibliographystyle{IEEEtran}
% \bibliography{btstack-manual}

% \section*{Bibliography}

%\textbf{P. W. Wachulak} received the degree${\ldots}$ \\[6pt]
%\textbf{M. C. Marconi} received the degree${\ldots}$ \\[6pt]
%\textbf{R. A. Bartels} received the degree${\ldots}$ \\[6pt]
%\textbf{C. S. Menoni} received the degree${\ldots}$ \\[6pt]
%\textbf{J. J. Rocca} received the degree${\ldots}$


\end{document}